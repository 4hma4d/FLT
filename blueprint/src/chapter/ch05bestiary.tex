\chapter{A collection of results which are needed in the proof.}

In this (temporary) chapter we list a whole host of definitions and theorems  which were known to humanity by the end of the 1980s and which we shall need. These definitions and theorems will find their way into more relevant sections of the blueprint once I have written more details. Note that some of these things are straightforward; others are multi-year research projects. The purpose of this chapter right now is to give the community some kind of idea of the task we face.

\section{Structures on the points of an affine variety.}

All rings and algebras in this section are commutative with a 1, and all morphisms send 1 to 1.

Let $X=\Spec(A)$ be an affine scheme of finite type over a field $K$. For example $X$ could be an affine algebraic variety; in fact we shall only be interested in smooth affine varieties in the applications, but the initial definition and theorem are fine for all finite type schemes.

If $R$ is any $K$-algebra then one can talk about the $R$-points $X(R)$ of $X$, which in this case
naturally bijects with the $K$-algebra homomorphisms from $A$ to $R$.

\begin{definition}\label{topology_on_affine_variety_points} If $X$ is an affine scheme of finite
    type over $K$, and if $R$ is a $K$-algebra which is also a topological ring, then we define a topology on the $R$-points $X(R)$ of $K$ by embedding the $K$-algebra homomorphisms from $A$ to $R$ into the set-theoretic maps from $A$ to $R$ with its product topology, and giving it the subspace topology.
\end{definition}

\begin{theorem}\label{topology_on_affine_variety_computation}
    If $X$ is as above and $X\to\mathbb{A}^n_K$ is a closed immersion, then the induced map from $X(R)$ with its topology as above to $R^n$ is an embedding (that is, a homeomorphism onto its image).
\end{theorem}
\begin{proof} See \href{https://math.stanford.edu/~conrad/papers/adelictop.pdf}{Conrad's notes}.
\end{proof}

We now specialise to the smooth case. I want to make the following conjectural ``definition'':
    
\begin{definition}\label{manifold_on_algebraic_variety_points} Let $K$ be a field equipped with an isomorphism to the reals, complexes, or a finite extension of the $p$-adic numbers. Let $X$ be a smooth affine algebraic variety over $K$. Then the points $X(K)$ naturally inherit the structure of a manifold over $K$.
\end{definition}

\begin{remark} Probably this is fine for a broader class of fields $K$. 
\end{remark}

\begin{conjecture}\label{manifold_on_algebraic_variety_computation} 
    If $X$ is as in the previous definition and $X\to\mathbb{A}^n_K$ is a closed immersion, then the induced map from $X(K)$ with its manifold structure to $K^n$ is an embedding. 
\end{theorem}

\begin{corollary}\label{lie_group_from_algebraic_group}
    If $G$ is an affine algebraic group of finite type over $K=\R$ or $\C$ then $G(K)$ is naturally a real or complex Lie group.
\begin{remark}

    The corollary, for sure, is true! And it's all we need. I have not yet made any serious effort to find a reference.

\section{Algebraic groups.}

The concept of an affine algebraic group over a field $K$ can be implemented in Lean as a commutative Hopf algebra over $K$, as a group object in the category of affine schemes over $K$, as a representable group functor on the category of affine schemes over $K$, or as a representable group functor on the category of schemes over $K$ which is represented by an affine scheme. All of these are the same to mathematicians
but different to Lean and some thought should go into which definition we use.

\begin{definition}\label{connected_reductive_group} An affine algebraic group over a field of characteristic zero is said to be \emph{connected} if it is connected as a scheme, and \emph{reductive} if it admits a finite-dimensional semisimple representation with finite kernel.
\end{definition}

\begin{remark} I have omitted characteristic $p$ (and in particular function fields over finite fields) only because I am less confident of the theory myself; probably these should not be omitted in the final formalised definition.
\end{remark}

\section{Automorphic forms and representations}

Let $G$ be a connected reductive group over a number field $N$. Let $\A_N^f:= N\otimes_{\Z}\widehat{Z}$
denote the finite adeles of $N$ and let $N_\infty := N\otimes_\Q\R$ denote the product of the completions of $N$ at the infinite places, so $\A_N:=\A_N^f\times N_\infty$ is the ring of adeles of $N$. We note
that $G(\A_N^f)$ is a (locally profinite) topological space and $G(N_\infty)$ is a real Lie group;
their product is $G(\A_N)$.

We are at some points vague in the below definition, whose details are complex. For FLT
we only need the definition for $G$ being either an abelian algebraic group, or an inner
form of $GL(2)$. Furthermore, we seem to need to fix a choice of maximal compact subgroup
$U_\infty$ of $G(N_\infty)$.

\begin{definition} An \emph{automorphic form} is a function $\phi:G(\A_N)\to\C$ satisfying the following conditions:
    \begin{itemize}
        \item $\phi$ is left-invariant under $G(N)$;


**TODO** write this properly.

Mention Mazur's theorem.

Then: 

JL, 

mult 1, 

cyclic base change

automorphic induction from GL_1(quad extn) to GL_2

Galois rep associated to an auto rep, 

Definition of an automorphic representation for the units of a quaternion algebra over a totally real field (including situations where the algebra is split at one or two infinite places).

Shimura curves and Shimura surfaces, plus a description of their etale cohomology in terms of automorphic representations.

Classification of finite subgroups of PGL_2(F_p-bar)

Moret-Bailly

Artin symbol of local class field theory

Existence of solvable extension avoiding a global extension and with prescribed local behaviour

Poitou-Tate

local Tate duality

