\chapter{An example of an automorphic form}

\section{Introduction}

The key ingredient in Wiles' proof of Fermat's Last Theorem is a \emph{modularity lifting theorem},
sometimes called an $R=T$ theorem. For Wiles, the $R$ came from elliptic curves, the $T$ came from 
classical modular forms, and the fact that they're equal is basically the Shimura--Taniyama--Weil 
conjecture, now known as the Breuil--Conrad--Diamond--Taylor modularity theorem: any elliptic
curve over the rationals is modular.

At the heart of the proof we shall formalise is also an $R=T$ theorem, however the $T$ which we
shall use will be associated not to classical modular forms, but to spaces of more general
automorphic forms called quaternionic modular forms. Those of you who know something about
classical modular forms might well know that the groups $\SL_2(\R)$ and $\SL_2(\Z)$ are
intimately involved; these are the norm 1 units in the matrix rings $M_2(\R)$ and $M_2(\Z)$. 
In the theory of quaternionic modular forms, the analogous groups are the norm 1 units
in rings such as Hamilton's quaternions $\R\oplus\R i\oplus\R j\oplus\R k$, and subrings
such as $\Z\oplus\Z i\oplus\Z j\oplus\Z k$.

One of the main goals of the FLT project at the time of writing this sentence, is formalising
the \emph{statement} of the modularity lifting theorem which we shall use. So we are going
to need to develop the theory of quaternionic modular forms, which is rather different
to the theory of classical modular forms (for example, in the cases we need, the definition 
is completely algebraic; there are no holomorphic functions in sight, and the analogue
of the upper half plane in the quaternionic theory is a finite set of points).

We could just launch into the general theory over totally real fields, which will be the generality
which we'll need. But when I was a PhD student, I learnt about
these objects by playing with explicit examples. So, whilst not logically necessary for the proof,
I thought it would be fun, and perhaps also instructional, to compute a concrete example of a space 
of quaternionic modular forms. The process of constructing the example might even inform what kind
of machinery we should be developing in general. Let's begin by discussing the quaternion algebra 
we shall use.

\section{A quaternion algebra}

Let's define $D$ to be the quaternion algebra $\Q\oplus\Q i\oplus\Q j\oplus\Q k$. As a vector
space, $D$ is 4-dimensional over $\Q$ with $[1,i,j,k]$ giving a basis. It has a (non-commutative)ring structure,
with multiplication satisfying the usual quaternion algebra relations $i^2=j^2=k^2=ijk=-1$. 
You can think of~$D$ as an analogue of $2\times 2$ 
matrices with rational coefficients, hence its units $D^\times$ are an analogue of the
group $\GL_2(\Q)$.

We will also need an analogue of the group $\GL_2(\Z)$, which will come from an integral structure
on~$D$. We choose the most obvious one, namely the subring $\calO:=\Z\oplus\Z i\oplus\Z j\oplus\Z k$.
This is not a "maximal order" -- one can also throw in $(1+i+j+k)/2$ and get a bigger lattice, 
but we don't need to do that in this example.

In this chapter, we are going to compute a complex vector space which could be called something
like the ``weight 2 level 2 modular forms for $D^\times$''. The main result will be that this
space is 1-dimensional, and this claim will
boil down to the fact that every prime (and hence every natural) is the sum of four
natural squares.

Note that mathlib has modular forms, but it doesn't have enough complex analysis to deduce
that the space of modular forms of a given weight and level is finite-dimensional. If all
the `sorry`s in this chapter are completed
before mathlib gets the necessary complex analysis, then the first nonzero space of modular forms
to be proved finite-dimensional in Lean will be a space of quaternionic modular forms.

\section{$\Zhat$}

Classically automorphic forms were defined as functions on symmetric spaces (like the upper half
plane) which transformed well under the action of certain discrete groups (for example $\SL_2(\Z)$).
However such definitions became combinatorially problematic when generalised to number fields 
with nontrivial class group, because the classical theory needed a \emph{number} $p$ to define
the Hecke operator $T_p$, and in the case where $p$ was a non-principal prime ideal in a number
field, there was no appropriate number. One fix is to take disjoint unions of symmetric spaces
indexed by the ideal class group of the field in question, but it is easier to work adelically,
which is morally what we shall do. However we are able to avoid introducing the adeles
explicitly; we can work instead with the conceptually simpler object $\Zhat$, 
the profinite completion of $\Z$. So what is $\Zhat$? We offer a low-level definition of this object. 

Given an integer $z$, we can reduce it mod $N$ for every positive natural
number and get elements $z_N=\overline{z}\in\Z/N\Z$. These elements are not completely arbitrary
though -- they must satisfy some compatibility conditions. For example there can be no
positive integer $z$ such that $z_{10}=6$ and $z_2=1$, because $z_{10}=6$ tells us that
$z$ ends in a 6 when written in base 10, and in particular it's even, so $z_2$ must be~0.
The general rule: if $D\mid N$ then $z_D$ must be equal to image of $z_N$ under the natural 
ring homomorphism from $\Z/N\Z$ to $\Z/D\Z$. We say that a collection of elements 
$z_N\in\Z/N\Z$ is \emph{compatible} if it satisfies this rule.

\begin{definition}\label{ZHat}\lean{ZHat} The profinite completion $\Zhat$ of $\Z$ is the set of 
    all compatible collections $(z_N)_N$ of elements of $\Z/N\Z$.
\end{definition}

To get this node green, we need to fill in the (hopefully straightforward) proofs showing that the 
compatibility condition cuts out a subring of $\prod_{N\geq1}(\Z/N\Z).$

Examples of elements of $\Zhat$ are are given by integers, where we define $z_N$ to be $z$ 
mod $N$ for all $N$. This gives us a natural injection from $\Z$ to $\Zhat$. But $\Zhat$
is much much larger than $\Z$; it has the same cardinality as the reals in fact. 
Let's write down an explicit example of an element of $\Zhat$ which isn't obviously in $\Z$.

\begin{definition}
    \label{ZHat.e}
    \lean{ZHat.e}
    \uses{ZHat}
    % blue node
    The infinite sum $0!+1!+2!+3!+4!+5!+\cdots$ looks 
    like it makes no sense at all; it is the sum of an infinite series of large positive integers. 
    However, the sum is \emph{finite} modulo $N$ for every positive integer $N$, because 
    all the terms from $N!$ onwards are multiples of $N$ and thus are zero in $\Z/N\Z$.
    Thus it makes sense to define $e_N$ to be the value of the finite sum modulo $N$.
    Explicitly, $e_N=0!+1!+\cdots+(N-1)!$ modulo $N$.  
    We claim that the collection $(e_N)_N$ is an element of $\Zhat$; the proof just involves
    manipulation of finite sums. 
\end{definition}

\begin{lemma}
    \label{ZHat.e_not_in_Int}
    \lean{ZHat.e_not_in_Int}
    \uses{ZHat.e}
    \notready % I don't know a proof!
    The element $(e_N)_N$ of $\Zhat$ is not in $\Z$.
\end{lemma}
\begin{proof}
    I don't have a proof of this but it doesn't look too bad; we will not need it later on.
\end{proof}

Let's prove some basic facts about $\Zhat$.

\begin{lemma}
    \label{ZHat.torsionfree}
    \lean{ZHat.torsionfree}
    \uses{ZHat}
    If $0<N$ is an integer then multiplication by $N$ is injective on $\Zhat$.
\end{lemma}
\begin{proof}
    Suppose that $(z_i)_i\in\Zhat$ and $Nz=0$. This means that $Nz_i=0\in\Z/i\Z$ for all $i$.
    Let us fix an arbitrary positive integer~$j$; we need to prove that $z_j=0\in\Z/j\Z$.
    Consider the element $z_{Nj}\in\Z/Nj\Z$. By assumption, we have $Nz_{Nj}=0$, meaning that
    if we lift $z_{Nj}$ to an integer, we have $Nj\mid Nz_{Nj}$, and thus $j\mid z_{Nj}$.
    Thus by the compatibility assumption on the $z_i$ we have that $z_j\in\Z/j\Z$ is the
    mod~$j$ reduction of $z_{Nj}$ and hence is zero.
\end{proof}

Later on will need to understand what the multiples of~$N$ are in $\Zhat$.

\begin{lemma}
    \label{ZHat.multiples}
    \lean{ZHat.multiples} 
    The multiples of~$N$ in $\Zhat$ are precisely the compatible collections $(z_i)_i\in\Zhat$ 
    with $z_N=0$.
\end{lemma}
\begin{proof}
    Clearly $z_N=0$ is a necessary condition to be a multiple of~$N$. To see it is sufficient, 
    take a general $(z_i)\in\Zhat$ which is a multiple of~$N$, note that $z_N=0$, 
    and now define a new element $(y_j)_j$ of $\Zhat$
    by $y_j=z_{Nj}/N$. Just to clarify what this means: $z_{Nj}\in\Z/Nj\Z$ maps to $z_N=0$ in $\Z/N\Z$
    by the compatibility assumption, so it is in the subgroup $N\Z/Nj\Z$ of $\Z/Nj\Z$, 
    which is isomorphic (via "division by $N$") to the ring $\Z/j\Z$; this is how we construct 
    $y_j$. It is easily checked that the $y_j$ are compatible and that $Ny=z$.
\end{proof}

\section{More advanced remarks on $\Zhat$ versus $\Q$}

This section can be skipped on first reading.

People who have seen some more advanced algebra might recognise this construction of $\Zhat$
as being the profinite completion of the additive abelian group $\Z$, so it is a fundamental
object of mathematics in some sense. But usually, when building mathematics, after $\Z$ we 
go to $\Q$, a multiplicative localisation of $\Z$, and only complete after that (to get $\R$).
The process of ``completing before localising'' gives us a far more arithmetic completion
of $\Z$.

Even though $\Q$ is a divisible abelian group and hence its profinite completion vanishes,
we can still attempt to "locally profinitely complete it" by defining $\Qhat:=\Q\otimes_{\Z}\Zhat$. 
This object is more commonly known as the \emph{finite adeles} of $\Q$. More generally if $F$ is
any number field then $F\otimes_{\Z}\Zhat$ is the ring of finite adeles of $F$. To get to
the full ring of adeles of a number field~$F$ you need to take the product with the
ring of infinite adeles of $F$, which is $F\otimes_{\Q}\R$: some kind of universal
archimedean completion of $F$. I don't know a reference which develops the theory of adeles
in this way, so this is what we shall do here.

\section{$\Qhat$ and tensor products.}

We define the commutative ring $\Qhat$ via a tensor product construction; we use this as an
excuse to run through the basic theory of tensor products.

We define $\Qhat$ to be $\Q\otimes\Zhat$, the tensor product of $\Q$
and $\Zhat$. We could more accurately write $\Qhat=\Q\otimes_{\Z}\Zhat$, but all tensor products
in this chapter are taken over the ring $\Z$, so we omit the subscript.

The first thing to know about the tensor product $A\otimes B$ of two abelian groups $A$ and $B$
is that given elements $a\in A$ and $b\in B$ we can form the element $a\otimes_t b\in A\otimes B$. 
Elements of this form are called \emph{pure tensors}. In the literature this element is often 
written $a\otimes b$, but we shall follow {\tt mathlib}'s convention in using $A \otimes B$ 
for sets/types and $a \otimes_t b$ for elements/terms. Addition of pure tensors obeys 
the ``distributivity'' rules $a\otimes_t b_1+a\otimes_t b_2=a\otimes_t(b_1+b_2)$
and $a_1\otimes_t b+a_2\otimes_t b=(a_1+a_2)\otimes_t b$, but there is no rule which simplifies
a general sum $a\otimes_t b + c\otimes_t d$ into a pure tensor. Indeed, in general it is not the 
case that every element of a tensor product $A\otimes B$ is of the form $a\otimes_t b$; there can be
tensors which aren't pure. However every element of $A\otimes B$ is a finite sum of pure tensors, 
with the result that one can attempt to define additive maps from $A\otimes B$ by saying what they 
do on pure tensors, and then extending linearly.

Another thing worth understanding is that just like how rational numbers can be written as 
quotients of integers in several  ways (for example $1/2=2/4=3/6=\cdots$), a general pure tensor 
in $A\otimes B$ can be represented as $a\otimes_t b$ in many ways. For example, in $\Qhat$
we have $1\otimes_t 2=2\otimes_t 1$. A general rule for equality of pure tensors is that if 
$a\in A$ and $b\in B$ and $z\in\Z$, then $za\otimes_tb=a\otimes_tzb$; integers can move over the 
tensor symbol. 

As a consequence of all of this, whilst it is easy to write down a general ``formula''
$\sum_ia_i\otimes_t b_i$ for an element of $A\otimes B$, it is not so easy to write down a 
``canonical form'' for such an element; there are typically many ways to express an element
of $A\otimes B$ as a finite sum of pure tensors, no one ``better'' than the others. This gives
the whole theory a slightly mystical status, because we are dealing with objects which
we cannot ``write down in a unique way''. However this does not stop us doing mathematics
with such objects!

One nice property of $\Qhat$ is that every tensor \emph{is} pure; let's prove this now.

\begin{lemma}\label{Qhat.canonicalForm} Every element of $\Qhat:=\Q\otimes_{\Z}\Zhat$
can be written as $q\otimes_t z$ with $q\in\Q$ and $z\in\Zhat$.
\end{lemma}
\begin{proof} A proof I would write on the board would look like the following. Take a general 
    element of $\Qhat$; we know it can be expressed as a finite sum
    $\sum_i q_i\otimes_t z_i$ with $q_i\in\Q$ and $z_i\in\Zhat$. Now choose a large 
    positive integer $N$, the lowest common multiple of all the denominators showing up in the
    $q_i$, and then rewrite $\sum_i q_i\otimes_t z_i$ as $\sum_i \frac{n_i}{N}\otimes z_i$ with 
    $n_i\in\Z$. Now using the fundamental fact that $na\otimes_t b=a\otimes_t nb$ for $n\in\Z$,
    we can rewrite the sum as $\sum_i \frac{1}{N}\otimes_t n_i z_i$
    which is equal to the pure tensor $\frac{1}{N}\otimes(\sum_i n_i z_i)$.

    In Lean I would prove this using {\tt TensorProduct.induction\_on}, which quickly
    reduces us to the claim that the sum of two pure tensors is pure, which we can prove
    using the above technique whilst avoiding the general theory of finite sums. 
\end{proof}

Be careful though: just because every element of $\Qhat$ can be written as $q\otimes z$, this
representation may not be unique. For example $2\otimes 1=1\otimes 2$.

\begin{remark} There does seem to be a unique ``lowest terms'' simple tensor representative of an 
    element of $\Qhat$; it is of the form $\frac{1}{D}\otimes_t z$ where $D$ is as small as
    possible; I claim that a pure tensor of this form is in lowest terms when $z\in\Zhat$ 
    satifies $z_D\in(\Z/D\Z)^\times$. We do not seem to need this fact in this chapter, but it
    would still be interesting to formalise.
\end{remark}

If $A$ and $B$ are additive abelian groups then $A\otimes B$ is also an additive abelian group.
However if $A$ and $B$ are commutative rings, then $A\otimes B$ also inherits the structure
of a commutative ring, with $0=0\otimes_t 0$ and $1=1\otimes_t 1$. Pure tensors multiply in the 
obvious way: the product of $a_1\otimes_t b_1$ and $a_2\otimes_t b_2$ is $a_1a_2\otimes_t b_1b_2.$
There are ring homomorphisms $A\to A\otimes B$ and $B\to A\otimes B$ sending $a$ to $a\otimes_t 1$
and $b$ to $1\otimes_t b$. In general such maps are not injective, but in the case of
$\Qhat=\Q\otimes\Zhat$ both maps from $\Q$ and $\Zhat$ are inclusions.

\begin{lemma}\label{Qhat.injective_Rat}\notready The ring homomorphism $\Q\to\Qhat$ sending $q$ to $q\otimes_t 1$
    is injective.
\end{lemma}
\begin{proof} Will be something to do with torsionfreeness.
\end{proof}

\begin{lemma}\label{Qhat.injective_ZHat}\notready The ring homomorphism $\Zhat\to\Qhat$ sending 
    $z$ to $1\otimes_t z$ is injective.
\end{lemma}
\begin{proof} Will be something to do with torsionfreeness.
\end{proof}

We can thus identify $\Q=\Q\otimes\Z$ and $\Zhat=\Z\otimes\Zhat$ with subrings of $\Qhat=\Q\otimes\Zhat$.
Note that, being commutative rings, $\Q$ and $\Zhat$ both contain a copy of $\Z$ as a subring, and
the corresponding copies of $\Z$ in $\Qhat$ are equal; this is because $1\otimes a=a\otimes 1$
for all $a\in\Z$.

\section{Additive structure of $\Qhat$.}

Here we forget the ring structure on everything, and analyse $\Qhat$ as an additive
abelian group, and in particular how the subgroups $\Z$, $\Q$ and $\Zhat$ sit within it.

The two results we prove in this section are that $\Q\cap\Zhat=\Z$ and that $\Q+\Zhat=\Qhat$.
Using lattice-theoretic notation we could write these results as $\Q\sqcap\Zhat=\Z$ and
$\Q\sqcup\Zhat=\Qhat$.

\begin{lemma}\label{Qhat.rat_meet_zHat} The intersection of $\Q$ and $\Zhat$ in $\Qhat$ is $\Z$.
\end{lemma}
\begin{proof}
    Shouldn't be hard.
\end{proof}

\begin{lemma}\label{Qhat.rat_join_zHat} The sum of $\Q$ and $\Zhat$ in $\Qhat$ is $\Qhat$.
    More precisely, every elemenet of $\Qhat$ can be written as $q+z$ with $q\in\Q$ and $z\in\Zhat$, 
    or more precisely as $q\otimes_t 1+1\otimes_t z$.
\end{lemma}
\begin{proof}
    Shouldn't be hard.
\end{proof}



