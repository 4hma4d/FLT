\chapter{Elliptic curves, and the Frey Curve}

\section{Overview}

In the last chapter we explained how, given a counterexample to Fermat's Last Theorem we could construct a Frey curve, which is an elliptic curve with some interesting properties. In this chapter we give an overview of parts of the theory of the arithmetic of elliptic curves. The two main results of this chapter are firstly that the $p$-torsion $\rho$ in the Frey curve is ``hardly ramified'', and secondly that Mazur's result on the possible torsion of elliptic curves implies that $\rho$ must be irreducible. 

\section{The arithmetic of elliptic curves}

We give an overview of the results we need, citing the literature for proofs. Everything here is standard, and most
of it dates back to the 1970s or before.

\begin{theorem}\label{Elliptic_curve_p_torsion_size}\lean{EllipticCurve.n_torsion_card}\tangled\discussion{12345}
  Let $n$ be a positive integer, let $K$ be a separably closed
  field with $n$ nonzero in $K$, and let $E$ be an elliptic curve over $K$. Then the $n$-torsion $E(K)[n]$ 
  in the $K$-points of $E$ is a finite group of size $n^2$.
\end{theorem}
\begin{proof}
  There are several proofs in the textbooks. The proof we shall formalise is forthcoming work of David Angdinata; it follows the approach with division polynomials, and it will be part of his PhD thesis.
\end{proof}

This theorem actually tells us the structure of the $n$-torsion, because of the following
purely group-theoretic result:
\begin{lemma}\label{group_theory_lemma}
  If $n$ is a positive integer and $A$ is a finite
  abelian group of size $n^2$ and if for all $d\mid n$, the $d$-torsion in $A$ has size $d^2$, 
  then $A\cong (\Z/n\Z)^2$. 
\end{lemma}
\begin{proof}
  One proof (which doesn't sound like much fun to formalise) writes $A$ as $\prod_{i=1}^t(\Z/a_i\Z)$
  with $a_i\mid a_{i+1}$, uses $d=a_1$ to deduce $t=2$ and then uses $d=a_2$ to deduce $a_1=a_2$.
\end{proof}

\begin{corollary}\label{Elliptic_curve_p_torsion_2d}\lean{EllipticCurve.n_torsion_dimension}
  Let $n$ be a positive integer, let $K$ be a separably closed
  field with $n$ nonzero in $K$, and let $E$ be an elliptic curve over $K$. Then the $n$-torsion $E(K)[n]$ 
  in the $K$-points of $E$ is a finite group isomorphic to $(\Z/n\Z)^2$.
\end{theorem}
\begin{proof}
  This follows from the previous group-theoretic lemma~\ref{group_theory_lemma} and
  theorem~\ref{Elliptic_curve_p_torsion_size}.
\end{proof}

If $K$ is now any field, and $E/K$ is an elliptic curve, then for $L$ and $M$ two fields
which are $K$-algebras, and for $\phi:L\to M$ a $K$-algebra morphism, there is
an induced $\Z/n\Z$-module morphism $E(L)[n]\to E(M)[n]$, which is functorial (that is,
the identity map $L\to L$ induces the identity map on $E(L)[n]$, and composing $K$-algebra
morphisms and then applying the construction is the same as applying the construction
and then composing). From this it is a purely formal fact that for $L/K$ a Galois extension,
$\Gal(L/K)$ acts naturally on $E(L)[n]$. If $K=\Q$, $L=\overline{\Q}$, $n>0$ and $E$ is an elliptic
curve over $\Q$, this gives us a \emph{Galois representation} $\Gal(\Qbar/\Q)\to\GL_2(\Z/n\Z)$
coming from the $n$-torsion of $E(\Qbar)$.

A fundamental fact about this Galois representation is that its determinant is the
cyclotomic character.

\begin{theorem}\label{Elliptic_curve_det_p_torsion}\uses{Elliptic_curve_p_torsion_2d} If $E$ is an 
  elliptic curve over a field $K$, and $n>0$ is a positive integer which is nonzero in $K$, then the 
  determinant of the 2-dimensional representation of $\Gal(K^\sep/K)$ on $E[n]$ is the 
  mod $n$ cyclotomic character.
\end{theorem}
\begin{proof}
  This presumably should be done via the Weil pairing. I have not yet put any thought into a feasible way to go about this.
\end{proof}

\section{Good reduction}

If $E$ is now an elliptic curve over the field of fractions $K$ of a DVR $R$ with maximal ideal 
$\m$, then we say $E$ has \emph{good reduction} if $E$ has a model with coefficients in $R$ and
the reduction mod $\m$ is still non-singular. If $E$ is an elliptic curve over a number field $N$
then one says that $E$ has good reduction at a finite place $P$ of $N$ if the base extension of $E$
to the completion $N_P$ of $N$ at $P$ has good reduction.

\begin{example} If $E$ is the Frey curve $Y^2=X(X-a^p)(X+b^p)$ and $q$ is a prime
  not dividing $abc$ (and in particular $q>2$), then the reduction mod $q$ of this 
  equation is still a smooth
  plane cubic, because the three roots $0$, $a^p$ and $-b^p$ are distinct modulo $q$
  (note that the difference between $a^p$ and $-b^p$ is $c^p$). Hence the Frey curve
  has good reduction at $q$.
\end{example}

If $E$ is an elliptic curve over a number field $N$ and if $\rho$ is the representation
of $\Gal(\overline{N}/N)$ on the $n$-torsion of $E$ then the image of $\rho$ is finite,
so by the fundamental theorem of Galois theory the representation factors through an
injection $\Gal(L/N)\to\GL_2(\Z/n\Z)$ where $L/N$ is a finite Galois extension of
number fields. One says that $\rho$ is \emph{unramified} at a finite place $P$ of $N$
if $L/N$ is unramified at $P$.

At some point we will need a theory of finite flat group schemes over an affine base. Here
is a working definition.

\begin{definition}\label{finite_flat_group_scheme} If $R$ is a commutative ring, then
  a \emph{finite flat group scheme} over $R$ is the spectrum of a commutative Hopf algebra $H/R$ 
  which is finite and flat as an $R$-module.
\end{definition}

Some facts we will need are:

\begin{theorem}\label{good_reduction_implies_unramified} If $E$ is an ellipitic curve over a number 
  field $N$ and $E$ has good reduction at a finite place $P$ of $N$, and if furthermore 
  $n\not\in P$, then the Galois representation on the $n$-torsion of $E$ is unramified.
\end{theorem}
\begin{proof}
  One approach to this would be by developing the theory of finite flat group schemes
  and considering the $n$-torsion on a good integral model for $E$. I have not really thought
  through whether this is the best approach. 
\end{proof}

\begin{theorem}\label{good_reduction_implies_flat} If $E$ is an ellipitic curve over a number field 
  $N$ and $E$ has good reduction at a finite place $P$ of $N$ containing the prime number $p$, 
  then the Galois representation on the $p$-torsion of $E$ comes from a finite flat group scheme 
  over the integers of the completion $N_P$.
\end{theorem}
\begin{proof}
  Indeed, the kernel of the $p$-torsion on a good integral model is finite and flat.
  Checking this claim will involve a fair amount of work.
\end{proof}

\section{Multiplicative reduction}

If $E$ is an elliptic curve over the field of fractions $K$ of a DVR $R$ with maximal ideal $\m$,
then we say that $E$ has multiplicative reduction if there is an integral model of $E$
which reduces mod $R/\m$ to a plane cubic with only one singularity, which is an ordinary double point.
We say that the reduction is \emph{split} if the two tangent lines at the ordinary double point
are both defined over $R/\m$, and \emph{non-split} otherwise.

\begin{example} If $E$ is the Frey curve $Y^2=X(X-a^p)(X+b^p)$ and $q$ is an odd prime
  which does divide $abc$ then the reduction mod $q$ of this equation is smooth
  away from the point $(x,0)$ where $x$ is the double root of the cubic mod $q$,
  and has an ordinary double point at $(x,0)$. Hence the Frey curve has
  multiplicative reduction at $q$.
\end{example}

\begin{example} If $E$ is the Frey curve $Y^2=X(X-a^p)(X+b^p)$ associated to a Frey package
  then $E$ has multiplicative reduction at 2. For the change of variables $X=4X'$ and $Y=8Y'+4X'$
  changes the equation to $64Y'^2+64X'Y'=64X'^3+16X'^2(b^p-a^p-1)-4X'a^pb^p$ and, because $p\geq5$,
  $b$ is even and $a=3$ mod 4 the 64s cancel, given an equation which reduces mod 2 to
  $Y'^2+X'Y'=X'^3+cX'^2$ for some $c\in\{0,1\}$, a cubic which is smooth away from an ordinary 
  double point at $(0,0)$. Hence the Frey curve has multiplicative reduction at~2. Note
  that $c=0$ iff $E$ has split multiplicative reduction (which happens iff $a^p=7$ mod $8$).
\end{example}

In particular, the Frey curve is \emph{semistable} -- it has good or multiplicative
reduction at all primes.

The main thing we need about elliptic curves with multiplicative reduction over $p$-adic fields
is the uniformisation theorem, originally due to Tate. 

\begin{theorem}\label{Tate_curve_uniformisation} If $E$ is an elliptic curve over a field
  complete with respect to a nontrivial nonarchimedean (real-valued) norm $K$ amd if $E$ has split
  multiplicative reduction, then there is a Galois-equivariant injection
  $\overline{K}^\times/q^{\mathbb{Z}}\to E(\overline{K})$.
\end{theorem}
\begin{proof}
  See~\cite{silverman2}, Theorems V.3.1, Remark V.3.1.2 (we don't need surjectivity),
  and Theorem V.5.3.
\end{proof}

\begin{corollary}\label{multiplicative_reduction_torsion} If $E$ is an elliptic curve
  over a field $K$ complete with respet to a nontrivial nonarchimedean (real-valued) norm
  and with perfect residue field, and if $E$ has multiplicative reduction, then the $n$-torsion
  $E(K^\sep)[n]$ 

******************************THE BELOW IS UNSORTED IDEAS****

Let $\rho:\GQ\to\GL_2(\Z/p\Z)$ be the representation on the $p$-torsion of this curve. In this chapter we discuss some basic properties of this representation, used both by Mazur to prove that $\rho$ cannot be reducible and by Wiles to prove that it can't be irreducible.

\section{Hardly ramified representations}

We make the following definition (this is not in the literature but it is a useful concept for us). We discuss the meaning of some of the concepts involved afterwards.

\begin{definition}\label{hardly_ramified} Let $p\geq5$ be a prime. A representation $\rho: \GQ\to \GL_2(\Z/p\Z)$ is said to be \emph{hardly ramified} if it satisfies the following four axioms:
  \begin{enumerate}
  \item $\det(\rho)$ is the mod $p$ cyclotomic character;
  \item $\rho$ is unramified outside $2p$;
  \item The semisimplification of the restriction of $\rho$ to $\Gal(\Qbar_2/\Q_2)$ is unramified.
  \item The restriction of $\rho$ to $\GQp$ comes from a finite flat group scheme;
  \end{enumerate}
\end{definition}

We are interested in hardly ramified representations for two reasons. One is that by using some very deep theorems, we will be able to prove that all hardly ramified representations are \emph{potentially automorphic}, which will give us our first foothold into the world of modular forms.

%\begin{theorem}\label{i_am_not_going_to_give_any_uses} A hardly ramified mod $p$ Galois representation $\rho:\GQ\to\GL_2(\Z/p\Z)$ is \emph{potentially automorphic}, becoming automorphic over a (possibly non-solvable) totally real Galois extension $F/\Q$ which is disjoint from the subfield cut out by the kernel of $\rho$.
%\end{theorem}
% Thm 5.3 Taylor notes
%\begin{proof}
%  This proof is deep. There are three major steps.
%
%  1) We first use Moret-Bailly's theorem to find an F and an F-point $A/F$ on a moduli space of elliptic curves with $p$-torsion $\rho$ and $\ell$-torsion induced from a certain character.
%
%  2) We then lift the character to characteristic zero, and prove that the resulting theta series is modular (using for example a converse theorem). We deduce that $A[\ell]$ is modular.
%
%  3) We now use a modularity lifting theorem (due to Wiles/Taylor/Khare-Wintenberger/Skinner-Wiles/who?) to deduce that $A$ is modular. We immediately deduce that $A[p]$ is modular as required.
%\end{proof}

The second reason is the theorem which we want to discuss in this section:

\begin{theorem}\label{frey_curve_hardly_ramified}\uses{hardly_ramified} If $\rho$ is the Galois representation on the $p$-torsion of the Frey curve coming from a Frey package $(a,b,c,p)$, then $\rho$ is hardly ramified.
\end{theorem}

The theorem will follow from the results below. The first three are valid for all elliptic curves, the rest are specific to Frey curves.




\begin{theorem}\label{Elliptic_curve_quotient_by_finite_subgroup} If $E$ is an elliptic curve over a field $K$ of characteristic zero, and $p$ is a prime, and if $C\subseteq E[p]$ is a Galois-stable
  subgroup of order $p$, then there's an elliptic curve ``''$E/C$'' over $K$ and an isogeny of elliptic curves $E\to E/C$ with kernel precisely $K$.
\end{theorem}
\begin{proof}
  Brian Conrad suggested the following approach, applicable as well for abelian schemes $A\to S$ over a base.  Let $G$ be a finite locally free $S$-subgroup of $A$, say $G$ with constant rank $n > 0$ by working locally on the base, so $G$ is contained in $A[n]$.  Then $n: A \to A$ is the fppf quotient of the source by $A[n]$, so it expresses $A$ as an $A[n]$-torsor over itself.  The problem of building $A/G$ as an abelian scheme is then seen to be the “same” as that of constructing the quotient of this $A[n]$-torsor by the G-action.

  In other words, the problem them becomes one having nothing specific to do with abelian schemes, at the cost of working over a base (such as the original target $A$) even when $S$ was the spectrum of a field in the application. The question is now: for a finite locally free commutative $S$-group $H$ and a closed locally free $S$-subgroup $G$, build a reasonable quotient $H/G$. One approach is to look at the Cartier dual $H^\vee\to G^\vee$, show that it's faithfully flat, and then deduce that the Cartier dual of the kernel of this map does the job. Note that one input for this proof is the claim that inclusions of Hopf algebras over fields are flat, proved nicely in Waterhouse’s book.
\end{proof}
\begin{theorem}\label{Frey_curve_j} If $(a,b,c,p)$ is a Frey package then the $j$-invariant of the corresponding Frey curve is $2^8(C^2-AB)^3/A^2B^2C^2$.
\end{theorem}
\begin{proof}
  Apply the explicit formula (presumably already in mathlib)
\end{proof}

Let us now study the Galois representation on the $p$-torsion of a Frey curve.

\begin{theorem}\label{Frey_curve_mod_p_rep_at_good_primes} Let $(a,b,c,p)$ be a Frey package and let $\rho$ be
  the $p$-torsion in the corresponding Frey curve. Let $\ell$ be a prime satisfying the three
  conditions (1) $\ell\ne2$ (2) $\ell\ne p$ (3) $\ell\nmid abc$. Then $\rho$ is unramified at the prime $\ell$.
\end{theorem}
\begin{proof}
  The hypotheses guarantee that the Frey curve has good reduction at $\ell$, and hence its
  $p$-torsion also has good reduction at $\ell$.
\end{proof}

For primes dividing $abc$ we will need a theory of the Tate curve.



For primes $\ell\not\in\{2,p\}$ which do divide $abc$, the Frey curve has bad (multiplicative) reduction at $\ell$. However the miracle is that the $p$-torsion of the Frey curve does is unramified anyway.

\begin{theorem}\label{Frey_curve_mod_p_rep_at_bad_primes}

  Let $(a,b,c,p)$ be a Frey package. If $2\ne\ell\ne p$ is a prime dividing $abc$ then the 
  $p$-torsion $\rho$ in the associated Frey curve is unramified at $\ell$.
\end{theorem}
\begin{proof} The Frey curve has multiplicative reduction at $\ell$ and the $p$-torsion is unramified at $\ell$ by the theory of Tate uniformisation (this uses the fact that the $j$-invariant is a power of $p$ after an unramified extension, and an explicit computation of the $p$-torsion in the Tate curve).
\end{proof}

\begin{theorem}\label{Frey_curve_mod_p_rep_at_2} Let $\rho$ be the $p$-torsion in a $p$-Frey curve. Then the semisimplification of the restriction of $\rho$ to $\Gal(\Qbar_2/\Q_2)$ is unramified.
\end{theorem}
\begin{proof}
  Again this follows from the theory of Tate uniformisation, and the fact that the Frey curve has multiplicative reduction at 2.
\end{proof}

\begin{theorem}\label{Frey_curve_mod_p_rep_at_p} Let $\rho$ be the $p$-torsion in a $p$-Frey curve. Then the restriction of $\rho$ to $\GQp$ comes from a finite flat group scheme.
\end{theorem}
\begin{proof} The Frey curve either has good reduction at $p$ (case 1 of FLT) or multiplicative reduction at $p$ (case 2 of FLT). In the first case the $p$-torsion is unramified at $p$ by general theory, and in the second case the theory of the Tate curve shows that the $p$-torsion is (up to quadratic twist) an \emph{unramified} extension of the trivial character by the cyclotomic character.
\end{proof}