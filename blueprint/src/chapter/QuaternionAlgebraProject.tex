\chapter{Project: Quaternion algebras}\label{Quat_alg_project}

\section{Introduction and goal}

At the time of writing, {\tt mathlib} still does not have a proof that the space
of classical modular forms of a fixed weight and level is finite-dimensional. The main result
in this project is to prove that spaces of quaternionic modular forms are
finite-dimensional. We need this to control the Hecke algebras which we'll define
later on using these spaces.

Let's start with the definition of these spaces. We fix a totally real field $F$
(that is, a number field $F$ such that the image of every ring homomorphism $F\to\bbC$
is a subset of $\R$). We fix a quaternion algebra $D$ over $F$. This means
the following: $D$ is an $F$-algebra of dimension 4, the centre of $D$ is $F$,
and $D$ has no nontrivial two-sided ideals. Examples of quaternion algebras
would be 2 by 2 matrices $M_2(F)$ over $F$, or the $F$ version of Hamilton's quaternions,
namely $F\oplus Fi\oplus Fj\oplus Fk$ with the usual laws $i^2=j^2=k^2=-1$ and
$ij=-ji=k$.

The high-falutin' explanation of what is about to happen is that $D^\times$
can be regarded as a reductive algebraic group over $F$, and we are going to define spaces
of automorphic forms for this algebraic group. In general such a definition would
involve some analysis (for example modular forms are automorphic forms for the
algebraic group $\GL_2$ over $\Q$, and the definition of a modular form involves
holomorphic functions, which are solutions to the Cauchy--Riemann equations).
However let us now make the assumption that $D$ is
\emph{totally definite}, which means that for every field map $\tau:F\to\R$,
the base extension $D\otimes_{F,\tau}\R$ along $\tau$ (which is a quaternion algebra
over the reals) is isomorphic to Hamilton's quaternions
$\R\oplus \R i\oplus\R j\oplus\R k$ rather than the other quaternion algebra
over the reals, namely $M_2(\R)$. Under this assumption the associated symmetric space
is 0-dimensional, meaning that no differential equations are involved in the definition
of an automorphic form in this setting. As a consequence, the definition we're about to give
makes sense not just over the complex numbers but over any commutative ring $R$, which will
be crucial for us as we will need to think about, for example, mod~$p$ automorphic forms in this
setting.

Having made an assumption on $D$ which makes the theory far less technical, we will now
make it a little more technical by using the modern adelic approach to the theory,
so we also need to talk about the finite adeles of $F$. This is a huge topological ring
denoted $\A_F^\infty$, which has a cheap definition as $F\otimes_{\Z}\Zhat$, where $\Zhat$ is
the profinite completion of the integers, but which is often explained in the literature
as the restricted product $\prod'_{\mathfrak{p}}F_{\mathfrak{p}}$ over the completions
$F_{\mathfrak{p}}$ of $F$ at all prime ideals $\mathfrak{p}\subseteq\mathcal{O}_F$ of the
integers of $F$. This is the subset of $\prod_{\mathfrak{p}}F_{\mathfrak{p}}$ consisting
of elements which are in $\mathcal{O}_{F,\mathfrak{p}}$ for all but finitely many $\mathfrak{p}$.
Mathlib already has the finite adeles and the proof that they're a topological ring. If you are
not familiar with this kind of object, just remember that it is a commutative ring with a topology,
because this is all we shall need.

Let us now fix a \emph{weight}, a \emph{level} and a \emph{character}, and we will define
a space of automorphic forms for $D^\times$ of this weight, level and character. If you know something
about the theory of classical modular forms you will have seen these words used in that theory.
In the adelic set-up these words have slightly different interpretations. We will define automorphic
forms over $R$, an arbitrary commutative base ring. If you are thinking about analogies with
spaces of classical modular forms then you could imagine the case $R=\bbC$.

A \emph{weight} is a finitely-generated $R$-module~$W$ with an action of $D^\times$. For applications
to Fermat's Last Theorem we only need to consider the case where $W=R$ and the action is trivial,
but there is no harm in setting things up in more generality. The case $W=R$ corresponds (via
the highly non-trivial Jacquet-Langlands correspondence) to the case of modular forms of weight 2.

A \emph{level} is a compact open subgroup~$U$ of $(D\otimes_F\A_F^\infty)^\times$. These are plentiful.
The ring $D\otimes_F\A_F^\infty$ is a topological ring, and this fact is currently in the process
of being PRed to mathlib. Hence the units $(F\otimes_F\A_F^\infty)^\times$ of this ring are a topological
group. This group is locally profinite, and hence has many compact open subgroups; we will see
explicit examples later on.

A \emph{character} is a group homomorphism $\chi$ from $(\A_F^\infty)^\times$ to $R^\times$. For many
of the applications, $\chi$ can also be trivial, although we will crucially have to allow
certain nontrivial characters of $p$-power order when we are doing the patching argument needed
to prove the modularity lifting theorem which is the first main goal of this project.
We regard $\A_F^\infty$ as a subring of $D\otimes_F\A_F^\infty$ (which makes sense because
$F$ is a subring of $D$) and hence we can regard $(\A_F^\infty)^\times$ as a (central)
subgroup of $(D\otimes_F\A_F^\infty)^\times$.

We fix a base ring $R$, a weight $W$, a level $U$ and a character $\chi$.

\begin{definition}
  \lean{TotallyDefiniteQuaternionAlgebra.AutomorphicForm}
  \label{TotallyDefiniteQuaternionAlgebra.AutomorphicForm}
  \leanok
  An \emph{automorphic form} of weight $W$, level $U$ and character $\chi$ for $D$ is
  a function $f:(D\otimes_F\A_F^\infty)^\times\to W$ satisfying the following axioms:
  \begin{itemize}
    \item $f(\delta g)=\delta\cdot f(g)$ for all $\delta\in D^\times$ and $g\in (D\otimes_F\A_F^\infty)^\times$
    (this makes sense because $W$ has an action of $D^\times$).
    \item $f(gz)=\chi(z)f(g)$ for all $g\in (D\otimes_F\A_F^\infty)^\times$ and $z\in(\A_F^\infty)^\times$
    (this makes sense because $W$ is an $R$-module).
    \item $f(gu)=f(g)$ for all $g\in (D\otimes_F\A_F^\infty)^\times$ and $u\in U$.
  \end{itemize}
\end{definition}

Let $S_{W,\chi}(U;R)$ denote the space of automorphic forms of weight $W$, level $U$ and character
$\chi$. Two basis observations are

\begin{definition}
  \lean{TotallyDefiniteQuaternionAlgebra.AutomorphicForm.addCommGroup}
  \label{TotallyDefiniteQuaternionAlgebra.AutomorphicForm.addCommGroup}
  \uses{TotallyDefiniteQuaternionAlgebra.AutomorphicForm}
  Pointwise addition $(f_1+f_2)(g):=f_1(g)+f_2(g)$ makes $S_{W,\chi}(U;R)$ into an additive
  abelian group.
\end{definition}

\begin{definition}
  \lean{TotallyDefiniteQuaternionAlgebra.AutomorphicForm.module}
  \label{TotallyDefiniteQuaternionAlgebra.AutomorphicForm.module}
  \uses{TotallyDefiniteQuaternionAlgebra.AutomorphicForm,
    TotallyDefiniteQuaternionAlgebra.AutomorphicForm.addCommGroup}
  Pointwise scalar multiplication $(r\cdot f)(g):= r\cdot(f(g))$ makes
  $S_{W,\chi}(U;R)$ into an $R$-module.
\end{definition}

  Later on we are going to be defining Hecke operators on this space of quaternionic modular
  forms, but the point of this mini-project is the following goal:

\begin{theorem}
  \lean{TotallyDefiniteQuaternionAlgebra.AutomorphicForm.finiteDimensional}
  \label{TotallyDefiniteQuaternionAlgebra.AutomorphicForm.finiteDimensional}
  \uses{TotallyDefiniteQuaternionAlgebra.AutomorphicForm.module}
  If $R$ is a field~$K$ and the weight $W$ is a finite-dimensional $K$-vector space
  then the space $S_{W,\chi}(U;K)$ is a finite-dimensional $K$-vector space.
\end{theorem}

This is an analogue of the result that classical modular forms of a fixed
level, weight and character are finite-dimensional. In fact, by delicate results
of Jacquet and Langlands this result implies many cases of that classical claim,
although the Jacquet--Langlands theorem is much harder to prove than anything
here.

The finite-dimensionality theorem is in fact an easy consequence of the following finiteness claim:
\begin{theorem}
  \lean{TotallyDefiniteQuaternionAlgebra.finiteDoubleCoset}
  \label{TotallyDefiniteQuaternionAlgebra.finiteDoubleCoset}
  If $U\subseteq (D\otimes_F\A_F^\infty)^\times$ is a compact open subgroup,
  then the double coset space $D^\times\backslash(D\otimes_F\A_F^\infty)^\times/U$ is a
  finite set.
\end{theorem}

I (kmb) had always imagined that this latter finiteness statement was ``folklore'' or
``a standard consequence of results about automorphic forms'', but in John Voight's
book~\cite{voight} this is Main Theorem 27.6.14(b) and Voight calls it Fujisaki’s lemma.

Let's prove finite-dimensionality of the space of quaternionic modular forms
assuming Fujisaki's lemma.
\begin{proof}
  \proves{TotallyDefiniteQuaternionAlgebra.AutomorphicForm.finiteDimensional}
  \uses{TotallyDefiniteQuaternionAlgebra.finiteDoubleCoset}
  Choose a set of coset representative $g_1,g_2,\ldots,g_n$ for
  $D^\times\backslash(D\otimes_F\A_F^\infty)^\times/U$. My claim is that
  the function $S_{W,\chi}(U;K)\to W^n$ sending $f$ to $(f(g_1),f(g_2),\ldots,f(g_n))$
  is injective and $K$-linear, which suffices by finite-dimensionality of $W$.
  $K$-linearity is easy, so let's talk about injectivity.

  Say $f_1$ and $f_2$ are two elements of $S_{W,\chi}(U;K)$ which agree on
  each $g_i$. It suffices to prove that $f_1(g)=f_2(g)$ for all
  $g\in(D\otimes_F\A_F^\infty)^\times$. So say $g\in(D\otimes_F\A_F^\infty)^\times$,
  and write $g=\delta g_iu$ for $\delta \in D^\times$ and $u\in U$.
  Then $f_1(g)=f_1(\delta g_iu)=\delta\cdot f_1(g_i)$ (by hypotheses (i) and (iii)
  of the definition of $S_{W,\chi}(U;K)$), and similarly $f_2(g)=\delta\cdot f_2(g_i)$
  and because $f_1(g_i)=f_2(g_i)$ by assumption, we deduce $f_1(g)=f_2(g)$ as required.
\end{proof}
