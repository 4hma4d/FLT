\chapter{An overview of the proof}\label{ch_overview}

So far we have seen that, modulo Mazur's theorem (and various other things which will still take some work to formalise but which are much easier), Fermat's Last Theorem can be reduced
to the statement that there is no prime $\ell\geq 5$ and hardly-ramified
irreducible 2-dimensional Galois representation $\rho:\GQ\to\GL_2(\Z/\ell\Z)$.

In this chapter we give an overview of our strategy for proving this, and collect
various results which we will need along the way. Note that we no longer need to assume that $\rho$ comes from the $\ell$-torsion in an elliptic curve.

\section{Potential modularity.}

We will only speak about modularity for 2-dimensional representations of the
absolute Galois group of a totally real field $F$ of even degree over $\Q$, and in particular we will
never say that a representation of the absolute Galois group of $\Q$ is modular! What we will
mean by ``modular'' is ``associated to an automorphic representation of the
units of the totally definite quaternion algebra over $F$ ramified at no finite places''.
We can furthermore even demand that the infinity type is trivial, as these are the only
forms we shall need for FLT.

Assume we have a hardly-ramified representation~$\rho$ as above. Let $K$
be the number field corresponding to the kernel of~$\rho$. Our first claim
is that there is some totally real field $F$ of even degree, Galois over $\Q$,
unramified at $\ell$, and disjoint from $K$, such that $\rho|G_F$ is modular.
The proof of this is very long, and uses a host of machinery. For example:
\begin{itemize}
    \item Moret--Bailly's result~\cite{moret-bailly} on points on curves with prescribed
    local behaviour;
    \item several nontrivial results in global class field theory;
    \item the Jacquet--Langlands correspondence;
    \item The assertion that irreducible 2-dimensional mod $p$ representations induced from a character are modular (this follows from converse theorems);
    \item A modularity lifting theorem.
\end{itemize}

Almost everything here dates back to the the 1980s or before.
The exception is the modularity lifting theorem, which we now state explicitly.

\section{A modularity lifting theorem}

Suppose $\ell\geq5$ is a prime, that $F$ is a totally real field of even degree in which $\ell$
is unramified, and that~$S$ is a finite set of finite places of~$F$ not dividing~$\ell$. Write
$G_F$ for the absolute Galois group of~$F$.

If $v\in S$ then let $F_v$ denote the completion of~$F$ at~$v$, fix an inclusion $\overline{F}\to\overline{F_v}$,
let $\calO_v$ denote the integers of $F_v$ and $k(v)$ the residue field. Let $I_v\subset G_F$ denote the inertia
subgroup at~$v$. Local class field theory (or a more elementary approach) gives a map $I_v\to\calO_{F_v}^\times$
and hence a map $I_v\to k(v)^\times$. Let $J_v$ denote the kernel of this map.

Let $R$ be a complete local Noetherian $\Z_\ell$-algebra with finite residue field of characteristic $\ell$.
We will be interested in representations $\rho:G_F\to\GL_2(R)$ with the following four properties.
\begin{itemize}
    \item $\det(\rho)$ is the cyclotomic character;
    \item $\rho$ is unramified outside $S\cup\{\ell\}$;
    \item If $v\in S$ then $\rho(g)$ has trace equal to~2 for all $g\in J_v$;
    \item If $v\mid\ell$ is a place of $F$ then $\rho$ is flat at~$v$.
\end{itemize}

In the last bullet point, ``flat'' means ``projective limit of representations arising from
finite flat group schemes''. Let us use the lousy temporary notation ``$S$-good'' to denote
representations with these four properties.

Say $k$ is a finite extension of $\Z/\ell\Z$ and $\rhobar:G_F\to\GL_2(k)$ is continuous,
absolutely irreducible when restricted to $F(\zeta_\ell)$, and $S$-good. One can check
that the functor representing $S$-good lifts of $\rhobar$ is representable.

\begin{theorem}
    \label{modularity_lifting_theorem}
    \uses{Skinner_Wiles_CFT_trick,local_galois_coh_dim_two,local_galois_coh_top_degree,
      local_galois_coh_poincare,local_galois_coh_euler_poincare,local_galois_coh_tate_duality,
      automorphic_representation_local_decomposition,cuspidal_automorphic_representation,
      Galois_representation_from_automorphic_representation_on_GL_2_form, moret-bailly,
      local_galois_coh_finite}
      \notready
        If $\rhobar$ is modular of level $\Gamma_1(S)$ and $\rho:G_F\to\GL_2(\calO)$ is
        an $S$-good lift of $\rhobar$ to $\calO$, the integers of a finite extension of $\Q_\ell$,
        then $\rho$ is also modular of level $\Gamma_1(S)$.
\end{theorem}

Right now we are very far from even stating this theorem in Lean.

I am not entirely sure where to find a proof of this in the literature, although it has certainly been known to the experts for some time. Theorem 3.3 of~\cite{taylor-mero-cont} comes close, although it assumes that $\ell$ is totally split in $F$ rather than just unramified. Another near-reference is Theorem~5.2 of~\cite{toby-modularity}, although this assumes
the slightly stronger assumption that the image of $\rho$ contains $\SL_2(\Z/p\Z)$ (however it is well-known to the experts that this can be weakened to give the result we need). One reference for the proof is \href{https://math.berkeley.edu/~fengt/249A_2018.pdf}{Richard Taylor's 2018 Stanford course}.

\begin{proof} (Sketch)

The proof is a two-stage procedure and has a nontrivial analytic input. First one uses the Skinner--Wiles trick to reduce to the ``minimal case'', and this needs cyclic base change for $\GL(2)$ and also a characterisation
of the image of the base change construction; this seems to need a multiplicity one result, which (because of our
definition of ``modular'') will need Jacquet--Langlands as well.

In the minimal case, the argument is the usual Taylor--Wiles trick, using refinements due to Kisin and others.
\end{proof}

Given this modularity lifting theorem, the strategy to show potential modularity of $\rho$ is to use Moret--Bailly to find an appropriate totally real field $F$, an auxiliary prime $p$, and an auxiliary elliptic curve over $F$ whose mod $\ell$ Galois representation is $\rho$ and whose
mod $p$ Galois representation is induced from a character. By converse theorems (for example)
the mod $p$ Galois representation is associated to an automorphic representation of
$\GL_2/F$ and hence by Jacquet--Langlands it is modular. Now we use the
modularity lifting theorem to deduce the modularity of the curve over $F$ and hence
the modularity of the $\ell$-torsion.

\section{Compatible families, and reduction at 3}

We now use Khare--Wintenberger to lift $\rho$ to a potentially modular $\ell$-adic
Galois representation of conductor 2, and put it into an $\ell$-adic family using
the Brauer's theorem trick in \cite{blggt}. Finally we look at the 3-adic specialisation
of this family. Reducing mod 3 we get a representation which is flat at 3 and tame at 2,
so must be reducible because
of the techniques introduced in Fontaine's paper on abelian varieties over $\Z$ (an irreducible
representation would cut out a number field whose discriminant violates the Odlyzko bounds).
One can now go on to deduce that the 3-adic representation must be reducible, which
contradicts the irreducibility of $\rho$.

We apologise for the sketchiness of what is here, however at the time of writing it is so far from what we are even able to \emph{state} in Lean that there seems to be little point right now in fleshing out the argument further. As this document grows, we will add a far more detailed discussion of what is going on here. Note in particular that stating the modularity lifting theorem in Lean is the first target.
