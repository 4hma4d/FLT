\chapter{An overview of the proof}\label{ch_overview}

So far we have seen that, modulo Mazur's theorem, Fermat's Last Theorem can be reduced
to the statement that there is no prime $\ell\geq 5$ and hardly-ramified
irreducible 2-dimensional Galois representation $\rho:\GQ\to\GL_2(\Z/\ell\Z)$.

In this short chapter we explain our strategy for proving this. This short chapter is written
for experts who want an overview of the route we're taking.

We will only speak about modularity for 2-dimensional representations of the
absolute Galois group of a totally real field $F$ of even degree over $\Q$, and
what we mean by the term is "associated to an automorphic representation of the
units of the totally definite quaternion algebra over $F$ ramified at no finite places".
We can furthermore even demand that the infinity type is trivial, as these are the only
forms we shall need for FLT.

Assume we have a hardly-ramified representation~$\rho$ as above. Let $K$
be the number field corresponding to the kernel of~$\rho$. Our first claim
is that there is some totally real field $F$ of even degree, Galois over $\Q$,
unramified at $\ell$, and disjoint from $K$, such that $\rho|G_F$ is modular. 
The proof of this is very long, and uses a host of machinery. For example:
\begin{itemize}
    \item Moret--Bailly's result~\cite{moret-bailly} on points on curves with prescribed 
    local behaviour;
    \item several nontrivial results in global class field theory;
    \item the Jacquet--Langlands correspondence;
    \item A modularity lifting theorem.
\end{itemize}

Everything here is from the 20th century and standard, other than the modularity
lifting theorem, which is explained in \href{https://math.berkeley.edu/~fengt/249A_2018.pdf}
{Taylor's 2018 Stanford course}. The strategy is to use Moret--Bailly to find an auxiliary 
elliptic curve over $F$ whose mod $\ell$ Galois representation is $\rho$ and whose
mod $p$ Galois representation is induced from a character. By converse theorems (for example)
the mod $p$ Galois representation is associated to an automorphic representation of
$\GL_2/F$ and hence by Jacquet--Langlands it is modular. Now we use the
modularity lifting theorem to deduce the modularity of the curve and hence
the modularity of the $\ell$-torsion. 

We now use Khare--Wintenberger to lift $\rho$ to a potentially modular $\ell$-adic
Galois representation of conductor 2, and put it into an $\ell$-adic famiily using
the Brauer's theorem trick in \cite{blggt}. Finally we look at the 3-adic specialisation
of this family. Reducing mod 3 we get a representation which must be reducible because
of the techniques introduced in Fontaine's paper on abelian varieties over $\Z$ (an irreducible
representation would cut out a number field whose discriminant violates the Odlyzko bounds).
One can now go on to deduce that the 3-adic representation must be reducible, which
contradicts the irreducibility of $\rho$.

As this document grows, we will be able to add links to a more detailed discussion of
what is going on here.


