\chapter{Automorphic forms and the Langlands Conjectures}

This chapter came from discussions between Patrick, Mario and myself, all currently
visiting the Hausdorff Research Institute for
Mathematics in Bonn. The ultimate goal is to formally state some version of the global Langlands
reciprocity conjectures for $\GL_n$ over $\Q$.

\section[Definition of an automorphic form]{Definition of an automorphic form for $\GL_n$ over $\Q$.}

The global Langlands reciprocity conjectures relate automorphic forms to Galois representations.
The statements for a general connected reductive group involve the construction of the Langlands
dual group, and we do not have quite enough Lie algebra theory to push this definition through
in general. However if we restrict the special case of the group $\GL_n/\Q$, the dual group
is just $\GL_n(\bbC)$ and several other technical obstructions are also removed. In this
section we will explain the definition of an automorphic form for the group $\GL_n/\Q$, following
the exposition by Borel and Jacquet in Corvallis.

\section{The finite adeles of the rationals.}

Mathlib already has the definition of the finite adeles $\A_{\Q}^f$ of the rationals as a
commutative $\Q$-algebra. It does not yet have the topology; work on this is on PR 13703 to mathlib.
See also the prerequisite PR 13705, which is ready for review. Note that 13703 still contains a
{\tt sorry}, but the result is true and not too hard. I tried proving it directly though and it ended
up a bit long. I propose the following plan. Start by proving the following induction principle for
finite adeles:
\begin{lemma}
    \label{DedekindDomain.FiniteAdeleRing.mul_induction_on}
    \lean{DedekindDomain.FiniteAdeleRing.mul_induction_on}
    \leanok
    If we have a predicate $P$ on the finite adeles, with the following properties:
  \begin{enumerate}
    \item $P(a)$ is true for $a$ any finite integral adele;
    \item $P(a)$ and $P(b)$ implies $P(ab)$;
    \item For every valuation $v$, $P(a)$ is true for every finite adele which is integral
      away from $v$ and is a unit times the inverse of a uniformiser at $v$.
  \end{enumerate}
  Then $P$ is true for all finite adeles.
\end{lemma}
\begin{proof}
  Every finite adele is integral away from a finite bad set $v\in S$. For these $v$,
  embed the local field $K_v$ into the finite adeles by sending $x_v\in K_v$ to the
  finite adele which is $x_v$ at $v$ and 1 elsewhere.

  If $\varpi_v$ is the finite adele corresponding a uniformiser at $v$,
  $a_v=u_v(\pi_v^{-1})^{n_v}$ for some unit $u_v$ and positive $n_v$ (the additive valuation
  of the integer $a_v^{-1}$). Note that $P(u_v)$ by hypothesis~1, and
  $P(\pi_v^{-1})$ by hypothesis~3.

  Now if $b=(b_v)_v$ is the finite adele with $b_v=a_v$ for $v\not\in S$ and $b_v=1$ for $v\in S$,
  then $b$ is a finite integral adele (so hypothesis 1 applies), and $a=b\prod_{v\in S}a_v$ is hence
  a finite product of finite adeles which satisfy $P$ and thus satisfies $P$ by hypothesis 2
  (with hypothesis 1 to deal with the base case $a=1$).
\end{proof}

Then prove the theorem.

\begin{theorem}
  \label{DedekindDomain.FiniteAdeleRing.clear_denominator}
  \lean{DedekindDomain.FiniteAdeleRing.clear_denominator}
  \leanok
Let $R$ be a Dedekind domain with field of fractions $K$. Then every $x\in\A_K^f$ can
be written as $x=st$ with $s\in\R\backslash\{0\}$ and $t\in\prod_v R_v$ a finite integral adele.
\end{theorem}
\begin{proof}
  \uses{DedekindDomain.FiniteAdeleRing.mul_induction_on}
  Use the induction procedure. The first and second hypotheses are easily checked
  (we can let $s=1$ for the first, and take the product of the $s$s for the second).
  For the third, if $a_v$ is integral then we use the first hypothesis,and if the
  additive valuation of $a_v$ is $-n<0$, we let $s$ be the $n$th power of a nonzero
  element in the prime ideal of $R$ corresponding to $v$.
\end{proof}

\begin{corollary}
  \label{DedekindDomain.instTopologicalRingFiniteAdeleRing}
  \lean{DedekindDomain.instTopologicalRingFiniteAdeleRing}
  \leanok
  The finite adeles are a topological ring.
\end{corollary}
\begin{proof} All this is done in the slightly technical mathlib PR 13703.
  \uses{DedekindDomain.FiniteAdeleRing.clear_denominator}
  \leanok
\end{proof}

\section[The adelic general linear group]{The group $\GL_n(\A_{\Q})$.}

The adeles $\A_{\Q}$ of $\Q$ are the product $\A_{\Q}^f\times\R$, with the product topology.
They are a topological ring. Hence $\GL_n(\A_{\Q})=\GL_n(\A_{\Q}^f)\times\GL_n(\R)$ is a
topological group, where we are being a bit liberal with our use of the equality symbol.

\section{Smooth functions}

\begin{definition}
  \label{AutomorphicForm.GLn.IsSmooth}
  \lean{AutomorphicForm.GLn.IsSmooth}
  \uses{DedekindDomain.instTopologicalRingFiniteAdeleRing}
  A function $f:\GL_n(\A_{\Q}^f)\times\GL_n(\R)\to\bbC$ is \emph{smooth}
  if it has the following three properties.

  \begin{enumerate}
    \item $f$ is continuous.
    \item For all $x\in\GL_n(\A_{\Q}^f)$, the function $y\mapsto f(x,y)$ is smooth.
    \item For all $y\in\GL_n(\R)$, the function $x\mapsto f(x,y)$ is locally constant.
  \end{enumerate}
\end{definition}

Current state of this definition: I've half-formalised it; I don't know how to say the the
function is smooth on the infinite part, because I have never used the manifold library before
and I have no idea what my model with corners is supposed to be.

\section{Slowly-increasing functions}

Automorphic representations satisfy a growth condition which we may as well factor out
into a separate definition.

We define the following temporary ``size'' function $s:\GL_n(\R)\to\R$ by
$s(M)=trace(MM^T+M^{-1}M^{-T})$ where $M^{-T}$ denotes inverse-transpose. Note that
$s(M)$ is always positive, and is large if $M$ has a very large or very small
(in absolute value) eigenvalue.

\begin{definition}
  \label{AutomorphicForm.GLn.IsSlowlyIncreasing}
  \lean{AutomorphicForm.GLn.IsSlowlyIncreasing}
  \leanok
  We say that a function $f:\GL_n(\R)\to\bbC$ is \emph{slowly-increasing}
  if there's some real constant $C$ and positive integer $n$ such that $|f(M)|\leq Cs(M)^n$
  for all $M\in\GL_n(\R)$.
\end{definition}

Note: the book says $n$ is positive, but $\{M|s(M)\leq 1\}$ is compact so I don't
think it makes any difference.

\section{Weights at infinity}

\begin{definition}
  \label{AutomorphicForm.GLn.weight}
  \lean{AutomorphicForm.GLn.weight}

The \emph{weight} of an automorphic form for $\GL_n/\Q$ can be thought of as a finite-dimensional
continuous complex representation $\rho$ of a maximal compact subgroup of $\GL_n(\R)$,
and it's convenient to choose one (they're all conjugate) so we choose $O_n(\R)$.
\end{definition}

The Lean definition is incomplete right now -- I don't demand irreducibility
(I wasn't sure whether I was doing this the right way; if I used category theory
then I might have struggled to say that the representation was continuous).


\section{The action of the universal enveloping algebra.}

\begin{definition}
  \label{instLieAlgebraAction}
There is a natural action of the real Lie algebra of $\GL_n(\R)$ on the complex vector space of
smooth complex-valued functions on $\GL_n(\R)$.
\end{definition}

\begin{definition}
  \label{instComplexLieAlgebraAction}
  \uses{instLieAlgebraAction}

  This extends to is a natural complex Lie algebra action of the complexification of
  the real Lie algebra, on the smooth complex functions on $\GL_n(\R)$.

\end{definition}

\begin{definition}
  \label{instUniversalEnvelopingAlgebraAction}
  \uses{instComplexLieAlgebraAction}
By functoriality, we get an action of the universal enveloping algebra of this
complexified Lie algebra on the smooth complex functions.

\end{definition}

\begin{definition}
\label{instCentreAction}
\uses{instUniversalEnvelopingAlgebraAction}
Thus the \emph{centre} $\Z_n$ of this universal enveloping algebra also acts on the smooth
complex functions.

\end{definition}

\begin{remark} The centre we just defined is a commutative ring which contains a copy of $\bbC$.
  Note that Harish-Chandra, or possibly this was known earlier, showed that it is a polynomial
  ring in $n$ variables over the complexes. We shall not need this.
\end{remark}

\section{Automorphic forms}

From here on there is no more Lean right now, only LaTeX.

\begin{definition}
  \label{AutomorphicForm.GLn.AutomorphicFormForGLnOverQ}
  \lean{AutomorphicForm.GLn.AutomorphicFormForGLnOverQ}
  \uses{AutomorphicForm.GLn.IsSmooth, AutomorphicForm.GLn.IsSlowlyIncreasing,
    AutomorphicForm.GLn.weight, instCentreAction}
  A smooth function $f:\GL_n(\A_{\Q}^f)\times\GL_n(\R)\to\bbC$ is
  an $O_n(\R)$-\emph{automorphic form} on $\GL_n(\A_{\Q})$ if it satisfies the following
  five conditions.
  \begin{enumerate}
    \item (periodicity) For all $g\in\GL_n(\Q)$, we have $f(gx,gy)=f(x,y)$.
    \item (has a finite level) There exists a compact open subgroup $U\subseteq\GL_n(\A_{\Q}^f)$
      such that $f(xu,y)=f(x,y)$ for all $u\in U$, $x\in \GL_n(\A_{\Q}^f)$ and $y\in\GL_n(\R)$.
    \item (weight $\rho$) There exists a continuous finite-dimensional irreducible complex
    representation $\rho$ of $O_n(\R)$ such that for every $(x,y)\in\GL_n(\A_{\Q})$, the
    set of functions $k\mapsto f(x,yk)$ span a finite-dimensional complex vector space isomorphic
    as $O_n(\R)$-representation to a direct sum of copies of $\rho$.
    \item (has an infinite level) There is an ideal $I$ of the centre $Z_n$ described in the
    previous section, which has finite complex codimension, and which annihiliates the
    function $y \mapsto f(x,y)$ for all $x\in \GL_n(\A_{\Q}^f)$. Note that this is a very fancy
    way of saying ``the function satisfies some natural differential equations''. In the
    case of modular forms, the differential equations are the Cauchy-Riemann equations, which
    is why modular forms are holomorphic.
    \item (growth condition) For every $x\in\GL_n(\A_{\Q}^f)$, the function $y\mapsto f(x,y)$
  on $\GL_n(\R)$ is slowly-increasing.
  \end{enumerate}

\end{definition}

Automorphic forms of a fixed weight $\rho$ form a complex vector space, and if we also
fix the finite level $U$ and the infinite level $I$ then we get a subspace which is
finite-dimensional; this is a theorem of Harish-Chandra. There is also the concept
of a cusp form, meaning an automorphic form for which furthermore some adelic integrals
vanish.

\section{Hecke operators}

\begin{lemma} The group $\GL_n(\A_{\Q}^f)$ acts (on the left) on the space of automorphic forms
  for $\GL_n(\A_{\Q})$ by the formula $(g\cdot f)(x,y)=f(xg,y)$.
\end{lemma}
\begin{proof}
  This is obvious. Note that the conjugate of a compact open subgroup is still
  compact and open.
\end{proof}

A formal development of the theory of Hecke operators looks like the following.

Let $U$ be a fixed compact open subgroup of $\GL_n(\A_{\Q}^f)$, and let's also fix
a weight $\rho$, and let $M_\rho(n)$ denote the complex vector space of automorphic
forms for $\GL_n/\Q$ of weight $\rho$. The level $U$ forms $M_\rho(n,U)$ are just the $U$-invariants
of this space. If $g\in\GL_n(\A_{\Q}^f)$, then I
claim that the double coset space $UgU$ can be written as a \emph{finite} disjoint union
of single cosets $g_iU$; one way of sesing this is that the double coset space is certainly
a disjoint union of left cosets, but the double coset space is compact and the left cosets
are open.

Define the Hecke operator $T_g:M_\rho(n,U)\to M_\rho(n,U)$ by
$T_g(f)=\sum g_i\cdot f$.

\begin{lemma} This function is well-defined, i.e., it sends a $U$-invariant form to
  a $U$-invariant form which is independent of the choice of $g_i$.
\end{lemma}
\begin{proof} Easy group theory.
\end{proof}
