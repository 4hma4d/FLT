\chapter{Miniproject: Adeles}\label{Adele_miniproject}

\section{Status}

This is an active miniproject.

\section{The goal}

There are several goals to this miniproject.

\begin{enumerate}
  \item Define the adeles $\A_K$ of a number field~$K$ and
    give them the structure of a $K$-algebra (status: now in mathlib thanks to
    Salvatore Mercuri);
  \item Prove that $\A_K$ is a locally compact topological ring (status:
      \href{https://github.com/smmercuri/adele-ring_locally-compact}{
      also proved by Mercuri} but not yet in mathlib);
  \item Base change: show that if $L/K$ is a finite extension of number fields then the
    natural map $L\otimes_K\A_K\to\A_L$ is an isomorphism; (status: not
    formalized yet, but there is a plan -- see the project dashboard);
  \item Prove that $K \subseteq \A_K$ is a discrete subgoup and the quotient
    is compact (status: not formalized yet, but there is a plan -- see the project
    dashboard);
  \item Get this stuff into mathlib (status: (1) done, (2)--(4) not done).
\end{enumerate}

We briefly go through the basic definitions. Let $K$ be a number field.
Let $\Zhat=\projlim_{N\geq1}(\Z/N\Z)$ be the profinite completion of $\Z$,
equipped with the projective limit topology.

A cheap definition of the finite adeles $\A_K^\infty$ of $K$ is $K\otimes_{\Z}\Zhat$,
equipped with the $\Zhat$-module topology.
A cheap definition of the infinite adeles
$K_\infty$ of $K$ is $K\otimes_{\Q}\R$ with the $\R$-module topology (this is a
finite-dimensional $\R$-vector space so this is just the usual topology on $\R^n$).
A cheap definition of the adeles of $K$ is $\A_K^\infty\times K_\infty$ with
the product topology. This is a commutative topological ring.

However in the literature different definitions
are often given. The finite adeles of $K$ are usually defined in the books
as the so-called restricted product $\prod'_{\mathfrak{p}}K_{\mathfrak{p}}$ over the completions
$K_{\mathfrak{p}}$ of $K$ at all maximal ideals $\mathfrak{p}\subseteq\mathcal{O}_K$ of the
integers of $K$. Here the restricted product is the subset of $\prod_{\mathfrak{p}}K_{\mathfrak{p}}$
consisting of elements which are in the integers $\mathcal{O}_{K,\mathfrak{p}}$ of
$K_{\mathfrak{p}}$ for all but finitely many $\mathfrak{p}$. This is the definition given in
mathlib. Mathlib also has the proof that they're a topological ring;
furthermore the construction of the finite adeles in mathlib works for any
Dedekind domain (this was pointed out to me by Maria Ines de Frutos Fernandez; the adeles
are an arithmetic object, but the finite adeles are an algebraic object).

Similarly the infinite adeles of a number field~$K$
are usually defined as $\prod_v K_v$,
the product running over the archimedean completions of~$K$, and this is
the mathlib definition.

The adeles of a number field $K$ are the product of the finite and infinite
adeles, and mathlib knows that they're a $K$-algebra and a topological ring.

\section{Local compactness}

As mentioned above, Salvatore Mercuri has a complete formalisation of the proof
that the adele ring is locally compact as a topological space. His work is in
\href{https://github.com/smmercuri/adele-ring_locally-compact}{his own repo} which
I don't want to have as a dependency of FLT, because this work should all be
in mathlib.

\begin{theorem}
  \lean{NumberField.AdeleRing.locallyCompactSpace}
  \label{NumberField.AdeleRing.locallyCompactSpace}
  \discussion{253}
  \leanok
  The adeles of a number field are locally compact.
\end{theorem}
\begin{proof}
  See \href{https://github.com/smmercuri/adele-ring_locally-compact/blob/e8e34608c139ee95a1e21d9d24f138524196a2e1/AdeleRingLocallyCompact/NumberTheory/NumberField/AdeleRing.lean#L70}
  {this line} in Mercuri's repo. The proof is: a restricted product of topological spaces $X_i$
  over compact open subspaces $C_i$ is locally compact, because $\prod_i C_i$ is open and compact
  (we should have this result in mathlib in general rather than the ad hoc approach which we
  currently have for adeles).
  Hence $\A_K^\infty$ is locally compact.
  The vector space $\R^n$ is locally compact. Hence $K_{\infty}$ is locally compact.
  Hence their product $\A_K$ is locally compact.
\end{proof}

\section{Base change}

The ``theorem'' we want is that if $L/K$ is a finite extension of number fields,
then $\A_L=L\otimes_K\A_K$. This isn't a theorem though, this is actually a \emph{definition}
(the map between the two objects) and a theorem about
the definition (that it's an isomorphism). In fact the full claim is that it is both a homeomorphism
and an $L$-algebra isomorphism. Before we can prove the theorem, we need to make the
definition.

Recall that the adeles $\A_K$ of a number field is a product $\A_K^\infty\times K_\infty$
of the finite adeles and the infinite adeles. So our ``theorem'' follows immediately from
the ``theorem''s that $\A_L^\infty=L\otimes_K\A_K^\infty$ and $L_\infty=L\otimes_KK_\infty$.
We may thus treat the finite and infinite results separately.

\subsection{Base change for finite adeles}

As pointed out above, the theory of finite adeles works fine for Dedekind domains.
So in this subsection let~$A$ be a Dedekind domain. Recall that the \emph{height one spectrum}
of $A$ is the nonzero prime ideals of~$A$. Note that because we stick to the literature,
rather than to common sense, fields are Dedekind domains in mathlib, and the
height one spectrum of a field is empty. The reason I don't like allowing fields
to be Dedekind domains is that geometrically the definition of Dedekind
domain in the literature is ``smooth affine curve, or a point''. But many theorems in algebraic
geometry begin ``let $C$ be a smooth curve'', rather than ``let $C$ be a smooth curve or a point''.

Let $K$ be the field of fractions of $A$. If $v$ is in the height one spectrum of $A$,
then we can put the $v$-adic topology on $A$ and on $K$, and consider the completions
$A_v$ and $K_v$. The finite adele ring $\mathbb{A}_{A,K}^\infty$ is defined to be
the restricted product of the $K_v$ with respect to the $A_v$, as $v$ runs over
the height one spectrum of $A$.

Now let~$L/K$ be a finite separable extension, and let $B$ be the integral closure of~$A$ in~$L$.
If $w$ is a nonzero prime ideal of $B$ lying under $v$, a prime ideal of $A$, then we can put the
$w$-adic topology on $L$ and the $v$-adic topology on~$K$. Furthermore
we can equip $K$ with an additive $v$-adic valuation, that is,
a function also called $v$ fron $K$ to $\Z\cup\{\infty\}$ normalised so that if $\pi$ is a uniformiser
for $v$ then $v(\pi)=1$. Similarly we consider $w$ as a function from $L$ to $\Z\cup\{\infty\}$.
The next lemma explains how these valuations are related.

\begin{lemma} If $i:K\to L$ denotes the inclusion then for $k\in K$ we have
  $e\times w(i(k))=v(k)$, where $e$ is the ramification index of $w/v$.
  \label{IsDedekindDomain.HeightOneSpectrum.valuation_comap}
  \lean{IsDedekindDomain.HeightOneSpectrum.valuation_comap}
  \leanok
\end{lemma}
\begin{proof}
  \leanok
  Standard (and formalized).
\end{proof}

\begin{definition}
  \lean{IsDedekindDomain.HeightOneSpectrum.adicCompletionComapAlgHom}
  \label{IsDedekindDomain.HeightOneSpectrum.adicCompletionComapAlgHom}
  \uses{IsDedekindDomain.HeightOneSpectrum.valuation_comap}
  \leanok
  There's a natural continuous $K$-algebra homomorphism map $K_v\to L_w$. It is defined by completing
  the inclusion $K\to L$ at the finite places $v$ and $w$ (which can be done
  because the previous lemma shows that the map is uniformly continuous for the $v$-adic
  and $w$-adic topologies).
\end{definition}

Now say $v$ is in the height one spectrum of $A$. The map above induces a continuous
$K$-algebra homomorphism $K_v\to\prod_{w|v}L_w$, where the product runs over the height one
primes of $B$ which pull back to $v$.

\begin{theorem}
  \lean{IsDedekindDomain.HeightOneSpectrum.adicCompletionComapAlgEquiv}
  \label{IsDedekindDomain.HeightOneSpectrum.adicCompletionComapAlgEquiv}
  \uses{IsDedekindDomain.HeightOneSpectrum.adicCompletionComapAlgHom}
  The induced continuous $L$-algebra homomorphism $L\otimes_KK_v\to\prod_{w|v}L_w$ is an
  isomorphism (both algebraic and topological).
\end{theorem}
\begin{proof}
  We follow Theorem 5.12 on p21 of \href{https://math.berkeley.edu/~ltomczak/notes/Mich2022/LF_Notes.pdf}
  {these notes}. We may write $L=K(\alpha)$ as a finite separable extension is simple. Let $f(x)$
  be the minimum polynomial of $\alpha$. Factor $f(x)=f_1(x)f_2(x)\cdots f_r(x)$ into
  monic irreducibles $K_v[x]$; these are distinct by separability. We have $L=K[x]/(f)$
  so $L\otimes_KK_v=K_v[x]/(f)=K_v[x]/(\prod_i f_i)=\oplus_i K_v[x]/(f_i)$. Write
  $L_i=K_v[x]/(f_i)$. We need to show that the $L_i$ correspond naturally to the completions
  $L_w$ for $w|v$.

  First note that $[L_i:K_v]\leq [L:K]<\infty$ and so there's a unique extension of the $v$-adic
  norm on $K_v$ to $L_i$. The restriction of this norm to $L$ must be equivalent to the $w$-adic
  norm for some $w|v$ by Ostrowski; then we must have $L_i=L_w$ because $L$ is dense in both
  and both are complete.

  Next note that if $i\not=j$ then $L_i$ and $L_j$ cannot be isomorphic as $L\otimes_KK_v$-algebras,
  because such an isomorphism would send $x$ to $x$ and thus show $f_i=f_j$. Hence the map
  from $L_i$ to the $w$ dividing $v$ is injective.

  For surjectivity, note that if $w|v$ then $L_w$ is an $L\otimes_KK_v$-algebra and hence
  admits a map from $L\otimes_K K_v$ which must factor through one of the $L_i$.
  This gives an injection $L_i\to L_w$. But $L_i$ is complete, as it's a finite extension
  of $K_v$, and the image is dense because $L$ is dense in $L_w$, hence $L_i=L_w$.
\end{proof}

\begin{theorem}
  \lean{IsDedekindDomain.HeightOneSpectrum.adicCompletionComapAlgEquiv_integral}
  \label{IsDedekindDomain.HeightOneSpectrum.adicCompletionComapAlgEquiv_integral}
  \uses{IsDedekindDomain.HeightOneSpectrum.adicCompletionComapAlgEquiv}
  The isomorphism $L\otimes_KK_v\to\prod_{w|v}L_w$ induces a topological isomorphism
  $B\otimes_AA_v\to \prod_{w|v}B_w$
  for all but finitely many $v$ in the height one spectrum of $A$.
\end{theorem}
\begin{proof}
  Certainly the image of the integral elements are integral. The argument in the other
  direction is more delicate. One approach (following Cassels--Froehlich, Cassels' article
  ``Global fields'', section 12 lemma, p61) is the following. Choose a $K$-basis
  $\omega_1,\omega_2,\ldots,\omega_n$ for $L/K$ with all $\omega_i\in A$.
  Then $\omega_1A\oplus\cdots\oplus\omega_nA$ and
  $B$ are two $A$-lattices in $L$, so they agree at almost all primes. In particular
  $\omega_1A_v\oplus\omega_2A_v\oplus\cdots\oplus\omega_nA_v=B\otimes_AA_v$ for all but
  finitely many primes $v$ of $A$. If we define the discriminant map~$D$ on $L/K$
  by $D(\gamma_1,\gamma_2,\ldots,\gamma_n)=det_{i,j}(trace_{L/K}(\gamma_i\gamma_j))$
  then it's well-known that $d:=D(\omega_1,\omega_2,\ldots,\omega_n)$ is nonzero (here we use
  separability), and hence a $v$-adic unit for all but finitely many $v$. Furthermore if
  $\gamma_i\in \prod_{w|v}B_w$ for all $i$ then $D(\gamma_1,\gamma_2,\ldots,\gamma_n)\in A_v$
  as all of the traces are in $A_v$. Now say
  we have an element of $\prod_{w|v}B_w$, and write it as $\sum_i b_i\omega_i$ with
  $b_i\in K_v$. Then for each $i$ we have
  $D(\omega_1,\omega_2,\ldots,\omega_{i-1},b,\omega_{i+1},\ldots,\omega_n)\in A_v$
  but it is also $b_i^2d$. Because $d$ is a $v$-adic unit for almost all $v$, we
  see that the $b_i$ must also be in $A_v$ for almost all $v$.
\end{proof}

We can take the product of the maps $K_v\to\prod_{w|v}L_w$ over $v$.

\begin{definition}
  \lean{DedekindDomain.ProdAdicCompletions.baseChange}
  \label{DedekindDomain.ProdAdicCompletions.baseChange}
  \uses{IsDedekindDomain.HeightOneSpectrum.adicCompletion_comap_algHom}
  There's a natural $K$-algebra homomorphism $\prod_v K_v\to\prod_w L_w$, where the
  products run over the height one spectra of $A$ and $B$ respectively.
\end{definition}

Note that we make no claim about continuity; such a claim wouldn't help
us because the adeles do not get the subspace topology.

\begin{theorem} This map induces a natural continuous $K$-algebra homomorphism
  $\A_{A,K}^\infty\to\A_{B,L}^\infty$.
  \label{DedekindDomain.FiniteAdeleRing.baseChange}
  \lean{DedekindDomain.FiniteAdeleRing.baseChange}
  \leanok
\end{theorem}
\begin{proof}
  This does not need the hard direction of
  theorem~\ref{IsDedekindDomain.HeightOneSpectrum.adicCompletionComapAlgEquiv_integral}.
  the definition of the map is straightforward using only that integers map
  to integers. Continuity also follows from this property; the induction function from
  $\prod_v A_v$ to $\prod_w B_w$ is continuous, and this gives continuity
  of the map at zero, and hence continuity everywhere.
\end{proof}

\begin{theorem}
  \label{DedekindDomain.FiniteAdeleRing.baseChangeEquiv}
  \lean{DedekindDomain.FiniteAdeleRing.baseChangeEquiv}
  \uses{IsDedekindDomain.HeightOneSpectrum.adicCompletionComapAlgEquiv_integral}
  \leanok
  If we give $L\otimes_K\A_{A,K}^\infty$ the ``module topology'', coming from the fact
  that $L\otimes_K\A_{A,K}^\infty$ is an $\A_{A,K}^\infty$-module, then the induced
  $L$-algebra morphism
  $L\otimes_K\A_{A,K}^\infty\to\A_{B,L}^\infty$ is a topological isomorphism.
\end{theorem}
\begin{proof}
  Existence of the map follows from
  theorem~\ref{IsDedekindDomain.HeightOneSpectrum.adicCompletionComapAlgEquiv_integral}.
  Surjectivity follows from
  theorem~\ref{IsDedekindDomain.HeightOneSpectrum.adicCompletionComapAlgEquiv_integral}.
  The fact that it is a topological isomorphism follows from the fact that
  $L\otimes_K K_v=\oplus_{w|v}L_w$ and that this identification identifies
  the subgroup $\calO_L\otimes_K K_v$ with $\oplus_{w|v}\calO_{L_w}$ for all
  but finitely many $v$.
\end{proof}

\subsection{Base change for infinite adeles}

Recall that if $K$ is a number field then the infinite adeles of $K$ are defined
to be the product $\prod_{v\mid\infty} K_v$ of all the completions of $K$ at the
infinite places.

The result we need here is that if $L/K$ is a finite extension of number fields,
then the map $K\to L$ extends to a continuous $K$-algebra map $K_\infty\to L_\infty$,
and thus to a continuous $L$-algebra isomorphism $L\otimes_KK_\infty\to L_\infty$.
Perhaps a cheap proof would be to deduce it from the fact that $K_\infty=K\otimes_{\Q}\R$.

\begin{theorem}
  \label{NumberField.InfiniteAdeleRing.baseChangeEquiv}
  \lean{NumberField.InfiniteAdeleRing.baseChangeEquiv}
  If $K\to L$ is a ring homomorphism between two number fields then there is a natural isomorphism
  (both topological and algebraic) $L\otimes_KK_\infty\cong L_\infty$.
\end{theorem}
\begin{proof}
  Standard.
\end{proof}

\subsection{Base change for adeles}

From the previous results we deduce immediately that if $L/K$ is a finite extension
of number fields then there's a natural (topological and algebraic) isomorphism
$L\otimes_K\A_K\to \A_L$.

\begin{theorem}
  \label{NumberField.AdeleRing.baseChangeEquiv}
  \lean{NumberField.AdeleRing.baseChangeEquiv}
  If $K\to L$ is a ring homomorphism between two number fields then there is a natural isomorphism
  (both topological and algebraic) $L\otimes_K\A_K\cong\A_L$.
\end{theorem}
\begin{proof}
  Follows from the previous results.
\end{proof}

Something else we shall need:

\begin{theorem}
  \label{NumberField.AdeleRing.baseChange_moduleTopology}
  If $K\to L$ is a ring homomorphism between two number fields then the topology on $\A_L$
  is the $\A_K$-module topology, where the module structure comes from the
  natural map $\A_K\to\A_L$.
\end{theorem}
\begin{proof}
  Because $\A_L=\A_K\otimes_{K}L$ we know that $\A_L$ is a finite free $\A_K$-module,
  and a standard fact (being PRed to mathlib) about the module topology is that the module topology
  on a finite free module is just the product topology. It thus suffices to show that
  the topology on $\A_L$ is the product topology after picking an $\A_K$-basis for $\A_L$
  but this is standard.
\end{proof}

\section{Discreteness and compactness}

We need that if $K$ is a number field then
$K\subseteq\mathbb{A}_K$ is discrete, and the quotient (with the
quotient topology) is compact. Here is a proposed proof.

\begin{theorem}
  \lean{Rat.AdeleRing.zero_discrete}
  \label{Rat.AdeleRing.zero_discrete}
  \leanok
  There's an open subset of $\A_{\Q}$ whose intersection with $\Q$ is $\{0\}$.
\end{theorem}
\begin{proof}
  Use $\prod_p{\Z_p}\times(-1,1)$. Any rational $q$ in this set is a $p$-adic
  integer for all primes $p$ and hence (writing it in lowest terms as $q=n/d$)
  satisfies $p\nmid d$, meaning that $d=\pm1$ and thus $q\in\Z$. The fact
  that $q\in(-1,1)$ implies $q=0$.
\end{proof}

\begin{theorem}
  \lean{Rat.AdeleRing.discrete}
  \label{Rat.AdeleRing.discrete}
  \leanok
  The rationals $\Q$ are a discrete subgroup of $\A_{\Q}$.
\end{theorem}
\begin{proof}
  If $q\in\Q$ and $U$ is the open subset in the previous lemma, then
  it's easily checked that $\Q\cap U=\{0\}$ implies $\Q\cap (U+q)=\{q\}$,
  and $U+q$ is open.
\end{proof}

\begin{theorem}
  \lean{NumberField.AdeleRing.discrete}
  \label{NumberField.AdeleRing.discrete}
  \leanok
  The additive subgroup $K$ of $\A_K$ is discrete.
\end{theorem}
\begin{proof}
  By a previous result, we have $\A_K=K\otimes_{\Q}\A_{\Q}$.
  Choose a basis of $K/\Q$; then $K$ can be identified with $\Q^n\subseteq(\A_{\Q})^n$
  and the result follows from the previous theorem.
\end{proof}

For compactness we follow the same approach.

\begin{theorem}
  \lean{Rat.AdeleRing.cocompact}
  \label{Rat.AdeleRing.cocompact}
  \leanok
  The quotient $\A_{\Q}/\Q$ is compact.
\end{theorem}
\begin{proof}
  The space $\prod_p\Z_p\times[0,1]\subseteq\A_{\Q}$ is a product of compact spaces
  and is hence compact. I claim that it surjects onto $\A_{\Q}/\Q$. Indeed,
  if $a\in\A_{\Q}$ then for the finitely many prime numbers $p\in S$ such that $a_p\not\in\Z_p$
  we have $a_p\in\frac{r_p}{p^{n_p}}+\Z_p$ with $r_p/p^{n_p}\in\Q$, and
  if $q=\sum_{p\in S}\frac{r_p}{p^{n_p}}\in\Q$ then $a-q\in \prod_p\Z_p\times\R$.
  Now just subtract $\lfloor a_{\infty}-q\rfloor$ to move into $\prod_p\Z_p\times[0,1)$
  and we are done.
\end{proof}

\begin{theorem}
  \lean{NumberField.AdeleRing.cocompact}
  \label{NumberField.AdeleRing.cocompact}
  \leanok
  The quotient $\A_K/K$ is compact.
\end{theorem}
\begin{proof}
  We proceed as in the discreteness proof above, by reducing to $\Q$. As before, choosing
  a $\Q$-basis of $K$ gives us $\A_K/K\cong(\A_{\Q}/\Q)^n$ so the result follows from
  the previous theorem.
\end{proof}
