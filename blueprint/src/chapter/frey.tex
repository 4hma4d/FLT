\chapter{The Frey Curve}

\section{Overview}

In the last chapter we explained how, given a counterexample to Fermat's Last Theorem we could construct a Frey curve, which is an elliptic curve with some interesting properties. Let $\rho:\GQ\to\GL_2(\Z/p\Z)$ be the representation on the $p$-torsion of this curve. In this chapter we discuss some basic properties of this representation, used both by Mazur to prove that $\rho$ cannot be reducible and by Wiles to prove that it can't be irreducible.

\section{Hardly ramified representations}

We make the following definition (this is not in the literature but it is a useful concept for us). We discuss the meaning of some of the concepts involved afterwards.

\begin{stealthdefinition} Let $p\geq5$ be a prime. A representation $\rho: \GQ\to \GL_2(\Z/p\Z)$ is said to be \emph{hardly ramified} if it satisfies the following four axioms:
  \begin{enumerate}
  \item $\det(\rho)$ is the mod $p$ cyclotomic character;
  \item $\rho$ is unramified outside $2p$;
  \item The semisimplification of the restriction of $\rho$ to is unramified.
  \item The restriction of $\rho$ to $\GQp$ comes from a finite flat group scheme;
  \end{enumerate}
\end{stealthdefinition}

The theorem we want to discuss in this section is:

\begin{theorem} If $\rho$ is the Galois representation on the $p$-torsion of the Frey curve coming from a Frey package, then $\rho$ is hardly ramified.
\end{theorem}
